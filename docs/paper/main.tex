%%%%%%%% ICML 2022 EXAMPLE LATEX SUBMISSION FILE %%%%%%%%%%%%%%%%%

\documentclass[nohyperref]{article}

% Recommended, but optional, packages for figures and better typesetting:
\usepackage{microtype}
\usepackage{graphicx}
\usepackage{subfigure}
\usepackage{booktabs} % for professional tables

% hyperref makes hyperlinks in the resulting PDF.
% If your build breaks (sometimes temporarily if a hyperlink spans a page)
% please comment out the following usepackage line and replace
% \usepackage{icml2022} with \usepackage[nohyperref]{icml2022} above.
\usepackage{hyperref}


% Attempt to make hyperref and algorithmic work together better:
\newcommand{\theHalgorithm}{\arabic{algorithm}}

% Use the following line for the initial blind version submitted for review:
% \usepackage{icml2023}

% If accepted, instead use the following line for the camera-ready submission:
\usepackage[accepted]{icml2023}

% For theorems and such
\usepackage{amsmath}
\usepackage{amssymb}
\usepackage{mathtools}
\usepackage{amsthm}
\usepackage{listings}

% if you use cleveref..
\usepackage[capitalize,noabbrev]{cleveref}

\definecolor{codegreen}{rgb}{0,0.6,0}
\definecolor{codegray}{rgb}{0.5,0.5,0.5}
\definecolor{codepurple}{rgb}{0.58,0,0.82}
\definecolor{backcolour}{rgb}{0.97,0.97,0.97}

\lstdefinestyle{mystyle}{
    backgroundcolor=\color{backcolour},
    commentstyle=\color{codegreen},
    keywordstyle=\color{magenta},
    numberstyle=\tiny\color{codegray},
    stringstyle=\color{codepurple},
    basicstyle=\ttfamily\footnotesize,
    breakatwhitespace=false,
    breaklines=true,
    captionpos=b,
    keepspaces=true,
    numbers=left,
    numbersep=5pt,
    showspaces=false,
    showstringspaces=false,
    showtabs=false,
    tabsize=2
}

\lstset{style=mystyle}

\DeclareMathOperator*{\argmax}{arg\,max}
\DeclareMathOperator*{\argmin}{arg\,min}

%%%%%%%%%%%%%%%%%%%%%%%%%%%%%%%%
% THEOREMS
%%%%%%%%%%%%%%%%%%%%%%%%%%%%%%%%
\theoremstyle{plain}
\newtheorem{theorem}{Theorem}[section]
\newtheorem{proposition}[theorem]{Proposition}
\newtheorem{lemma}[theorem]{Lemma}
\newtheorem{corollary}[theorem]{Corollary}
\theoremstyle{definition}
\newtheorem{definition}[theorem]{Definition}
\newtheorem{assumption}[theorem]{Assumption}
\theoremstyle{remark}
\newtheorem{remark}[theorem]{Remark}

% Todonotes is useful during development; simply uncomment the next line
%    and comment out the line below the next line to turn off comments
%\usepackage[disable,textsize=tiny]{todonotes}
\usepackage[textsize=tiny]{todonotes}


% The \icmltitle you define below is probably too long as a header.
% Therefore, a short form for the running title is supplied here:
\icmltitlerunning{Time interpret}

\begin{document}

\twocolumn[
\icmltitle{Time interpret package}

% It is OKAY to include author information, even for blind
% submissions: the style file will automatically remove it for you
% unless you've provided the [accepted] option to the icml2022
% package.

% List of affiliations: The first argument should be a (short)
% identifier you will use later to specify author affiliations
% Academic affiliations should list Department, University, City, Region, Country
% Industry affiliations should list Company, City, Region, Country

% You can specify symbols, otherwise they are numbered in order.
% Ideally, you should not use this facility. Affiliations will be numbered
% in order of appearance and this is the preferred way.
\icmlsetsymbol{equal}{*}

\begin{icmlauthorlist}
\icmlauthor{Joseph Enguehard}{comp}
\end{icmlauthorlist}

\icmlaffiliation{comp}{Babylon Health, 1 Knightsbridge Grn, London SW1X 7QA United Kingdom}

\icmlcorrespondingauthor{Joseph Enguehard}{joseph.enguehard@babylonhealth.com}

% You may provide any keywords that you
% find helpful for describing your paper; these are used to populate
% the "keywords" metadata in the PDF but will not be shown in the document
\icmlkeywords{Machine Learning, ICML}

\vskip 0.3in
]

% this must go after the closing bracket ] following \twocolumn[ ...

% This command actually creates the footnote in the first column
% listing the affiliations and the copyright notice.
% The command takes one argument, which is text to display at the start of the footnote.
% The \icmlEqualContribution command is standard text for equal contribution.
% Remove it (just {}) if you do not need this facility.

\printAffiliationsAndNotice{}  % leave blank if no need to mention equal contribution
% \printAffiliationsAndNotice{\icmlEqualContribution} % otherwise use the standard text.

\begin{abstract}
    This is the abstract.
\end{abstract}

\section{Introduction}
\label{sec:introduction}

As neural networks are becoming more relied on for many decision processes, there has been an increasing focus into
understanding how these algorithms come to a specific prediction.
Should such models be used to take high-stakes decisions, as it is often the case in finance or in medicine, there is an
increasing pressure on these tools to provide a justification along with their results, which could be used to discuss
or oppose a decision.
The risk is that deep learning models, considered as $``$black boxes$"$, could produce biased and unfair toward certain
categories of society.
Such behaviour has already been noticed in opaque algorithms such as COMPAS, used to evaluate the risk of recidivism,
which was accused of being biased toward people of color.

As a result, many methods aiming to explain a deep learning prediction have been developed, which now constitutes a
field called Explainable AI (XAI).
Among these methods, some of the most common are LIME, SHAP, or Integrated Gradients.
Alongside this research, there has also been a drive to create libraries unifying these different methods, enabling
their use on popular deep learning libraries, and integrating evaluation tools to compare these XAI methods.
To this end, several pieces of software have been proposed, including SHAP, InterpretML, OmniXAI or Captum, among many.

However, several researchers have noticed a lack of attention toward a specific type of data: time series.
Temporal data is nevertheless crucial in many applications: financial and medical data are commonly multivariate time
series.
Models which produce predictions based on this type of data need a careful consideration, as these applications often
carry high-stakes decisions.
Consequently, several feature attribution methods have been introduced to tackle this specific case.
Yet, there seems to be a lack of a unified library to regroup and evaluate these specific methods.

As a result, we created \texttt{\detokenize{time_interpret}} (short: \texttt{tint}), a Python library designed as an
extension of Captum~\citep{kokhlikyan2020captum}.
Although this library can be used with any PyTorch~\citep{NEURIPS2019_9015} model, it has a specific focus on time series,
providing several feature attribution methods developed for this specific type of data.
\texttt{\detokenize{time_interpret}} also provides evaluation tools, whether the true attributions are known or not, as
well as several time series datasets.
It also leverages PyTorch Lightning~\citep{Falcon_PyTorch_Lightning_2019} to simplify the use of the original PyTorch
library.
As such, it provides several common PyTorch models used to handle temporal data, as well as a specific PyTorch Lightning
wrapper.

Moreover, despite this focus on time series, several components of \texttt{\detokenize{time_interpret}} have a slightly
different application.
It provides for instance several methods aiming to explain language models such a BERT\@.
Its evaluation tools can also be used with any feature attribution methods, and not just the ones implemented in
this library.

This paper aim to give a general introduction to \texttt{\detokenize{time_interpret}}.
Furthermore, several previously unpublished methods have been developed along with this library, which we also present
here.
We hope this study will give more clarity to the corresponding codebase, and will prove useful for further research
in this field.
We encourage the reader to also refer to the library documentation for more information, especially in case of
new significant releases.
\section{Presentation of the library}
\label{sec:presentation}

We provide in this section an introduction to the \texttt{\detokenize{time_interpret}} library.
Please also refer to the documentation.

\texttt{\detokenize{time_interpret}} is primarily composed of 4 different parts: attribution methods, datasets,
evaluation tools (metrics) and deep learning models.
We present below an introduction as well as a short description of the components in each of these parts.


\paragraph{Attribution methods.}

Attribution methods constitutes the core of \texttt{\detokenize{time_interpret}}.
In this part of the library, we regrouped many methods which have been recently published.
Similarly to Captum~\citep{kokhlikyan2020captum}, each method can be called like this:

\begin{lstlisting}[language=Python, caption=Python example, label={lst:import}]
from tint.attr import TemporalIntegratedGradients

explainer = TemporalIntegratedGradients(
    model
)
attr = explainer.attribute(inputs)
\end{lstlisting}

where $``$model$"$ is a PyTorch model, and $``$inputs$"$ is an inputs' tensor.

We provide in this library several methods:

\begin{itemize}
    \item \textbf{AugmentedOcclusion}.
        This method improves upon the original Occlusion method from captum~\url{https://captum.ai/api/occlusion.html}
        by allowing to sample the baseline from a bootstrapped distribution.
        By selecting a distribution close to the inputs, the resulted occulted data should be close to actual data as a
        result, limiting the amount of out of distribution samples.
        This method was originally proposed by~\citep{tonekaboni2020went}, Section 4.
        Please refer to this paper for more details.
    \item \textbf{BayesLime, BayesKernelShap}.
        These two method, originally proposed by~\citep{slack2021reliable}, extend respectively
        LIME~\citep{ribeiro2016should} and KernelSHAP~\citep{lundberg2017unified}, by replacing the underlying
        linear regression model with a bayesian linear regression, allowing the method to model uncertainty in
        explainability, by outputting credible intervals.
    \item \textbf{DiscretetizedIntegratedGradients (DIG)}.
        DIG~\citep{sanyal2021discretized} was designed to interpret predictions made by language models.
        It builds upon the original Integrated Gradients method by generating discretized paths, hopping from one
        word to another, instead of using straight lines.
        This way, it aims to create a path which takes into account the discreteness of the embedding space.
    \item \textbf{DynaMask}.
        This method, introduced by~\citep{crabbe2021explaining}, is an adaptation of a perturbation-based method
        developed in~\citep{fong2017interpretable, fong2019understanding}, to handle time-series data.
        As such, it consists in perturbing a temporal data by replacing some of it with an average in time.
        The mask used to choose which data should be preserved and which should be replaced is learnt in order to either
        preserve the original prediction with a minimum amount of data, or change the original prediction with a small
        amount of data.
        Either way, the learnt mask can then be used to discriminate between important features and others.
    \item \textbf{ExtremalMask}.
        This method consists in a generalisation of DynaMask, which learns not only the mask, but also the associated
        perturbation, instead of replacing perturbed data with a predetermined average.
    \item \textbf{Fit}.
        Originally proposed by~\citep{tonekaboni2020went}, this method aims to understand which feature is important by
        quantifying the shift in the predictive distribution over time.
        An important feature is then one which contributes significantly to the distributional shift.
    \item \textbf{LofLime, LofKernelShap}.
        Novel method.
        Please see Section~\ref{sec:methods} for more details.
    \item \textbf{NonLinearitiesTunnel}.
        Novel method.
        Please see Section~\ref{sec:methods} for more details.
    \item \textbf{Retain}.
    \item \textbf{SequentialIntegratedGradients (SIG)}.
    \item \textbf{TemporalOcclusion}.
    \item \textbf{TemporalAugmentedOcclusion}.
    \item \textbf{TemporalIntegratedGradients}.
        Novel method.
        Please see Section~\ref{sec:methods} for more details.
    \item \textbf{TimeForwardTunnel}.
        Novel method.
        Please see Section~\ref{sec:methods} for more details.

\end{itemize}


\paragraph{Datasets}

\paragraph{Metrics}

\paragraph{Models}

\section{Novel methods}
\label{sec:methods}

This is the methods section.


\subsection{Temporal attribution methods}
\label{subsec:temporal-attribution-methods}

\paragraph{Temporal Integrated Gradients.}

\paragraph{Time forward tunnel.}


\subsection{Other attribution methods}
\label{subsec:other-attribution-methods}

\paragraph{Local Outlier Factor LIME and Kernel SHAP\@.}

\paragraph{Non-linearities tunnel.}

\section{Conclusion}
\label{sec:conclusion}

This is the conclusion.


% In the unusual situation where you want a paper to appear in the
% references without citing it in the main text, use \nocite
% \nocite{langley00}

\bibliography{bibliography}
\bibliographystyle{icml2023}


%%%%%%%%%%%%%%%%%%%%%%%%%%%%%%%%%%%%%%%%%%%%%%%%%%%%%%%%%%%%%%%%%%%%%%%%%%%%%%%
%%%%%%%%%%%%%%%%%%%%%%%%%%%%%%%%%%%%%%%%%%%%%%%%%%%%%%%%%%%%%%%%%%%%%%%%%%%%%%%
% APPENDIX
%%%%%%%%%%%%%%%%%%%%%%%%%%%%%%%%%%%%%%%%%%%%%%%%%%%%%%%%%%%%%%%%%%%%%%%%%%%%%%%
%%%%%%%%%%%%%%%%%%%%%%%%%%%%%%%%%%%%%%%%%%%%%%%%%%%%%%%%%%%%%%%%%%%%%%%%%%%%%%%
% \newpage
% \appendix
% \onecolumn
% \input{appendix}
%%%%%%%%%%%%%%%%%%%%%%%%%%%%%%%%%%%%%%%%%%%%%%%%%%%%%%%%%%%%%%%%%%%%%%%%%%%%%%%
%%%%%%%%%%%%%%%%%%%%%%%%%%%%%%%%%%%%%%%%%%%%%%%%%%%%%%%%%%%%%%%%%%%%%%%%%%%%%%%


\end{document}


% This document was modified from the file originally made available by
% Pat Langley and Andrea Danyluk for ICML-2K. This version was created
% by Iain Murray in 2018, and modified by Alexandre Bouchard in
% 2019 and 2021 and by Csaba Szepesvari, Gang Niu and Sivan Sabato in 2022.
% Previous contributors include Dan Roy, Lise Getoor and Tobias
% Scheffer, which was slightly modified from the 2010 version by
% Thorsten Joachims & Johannes Fuernkranz, slightly modified from the
% 2009 version by Kiri Wagstaff and Sam Roweis's 2008 version, which is
% slightly modified from Prasad Tadepalli's 2007 version which is a
% lightly changed version of the previous year's version by Andrew
% Moore, which was in turn edited from those of Kristian Kersting and
% Codrina Lauth. Alex Smola contributed to the algorithmic style files.
